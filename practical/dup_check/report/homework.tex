\documentclass[a4paper,twoside]{article}
\usepackage{blindtext}  
\usepackage{geometry}

% Chinese support
\usepackage[UTF8, scheme = plain]{ctex}

% Page margin layout
\geometry{left=2.3cm,right=2cm,top=2.5cm,bottom=2.0cm}


\usepackage{listings}
\usepackage{xcolor}
\usepackage{geometry}
\usepackage{amsmath}
\usepackage{float}
\usepackage{hyperref}

\usepackage{graphics}
\usepackage{graphicx}
\usepackage{subfigure}
\usepackage{epsfig}
\usepackage{float}

\usepackage{algorithm}
\usepackage[noend]{algpseudocode}

\usepackage{booktabs}
\usepackage{threeparttable}
\usepackage{longtable}
\usepackage{listings}
\usepackage{tikz}

% cite package, to clean up citations in the main text. Do not remove.
\usepackage{cite}

\usepackage{color,xcolor}

%% The amssymb package provides various useful mathematical symbols
\usepackage{amssymb}
%% The amsthm package provides extended theorem environments
\usepackage{amsthm}
\usepackage{amsfonts}
\usepackage{enumerate}
\usepackage{enumitem}
\usepackage{listings}

\usepackage{indentfirst}
\setlength{\parindent}{2em} % Make two letter space in the first paragraph
\usepackage{setspace}
\linespread{1.5} % Line spacing setting
\usepackage{siunitx}
\setlength{\parskip}{0.5em} % Paragraph spacing setting

% \usepackage[contents =22920202204622, scale = 10, color = black, angle = 50, opacity = .10]{background}

\renewcommand{\figurename}{图}
\renewcommand{\lstlistingname}{代码} 
\renewcommand{\tablename}{表格}
\renewcommand{\contentsname}{目录}
\floatname{algorithm}{算法}

\graphicspath{ {images/} }

%%%%%%%%%%%%%
\newcommand{\StudentNumber}{22920202204622}  % Fill your student number here
\newcommand{\StudentName}{熊恪峥}  % Replace your name here
\newcommand{\PaperTitle}{实验报告(二)查重}  % Change your paper title here
\newcommand{\PaperType}{算法设计与分析} % Replace the type of your report here
\newcommand{\Date}{2022年3月23日}
\newcommand{\College}{信息学院}
\newcommand{\CourseName}{算法设计与分析}
%%%%%%%%%%%%%

%% Page header and footer setting
\usepackage{fancyhdr}
\usepackage{lastpage}
\pagestyle{fancy}
\fancyhf{}
% This requires the document to be twoside
\fancyhead[LO]{\texttt{\StudentName }}
\fancyhead[LE]{\texttt{\StudentNumber}}
\fancyhead[C]{\texttt{\PaperTitle }}
\fancyhead[R]{\texttt{第{\thepage}页,共\pageref*{LastPage}页}}


\title{\PaperTitle}
\author{\StudentName}
\date{\Date}

\lstset{
	basicstyle          =   \sffamily,          % 基本代码风格
	keywordstyle        =   \bfseries,          % 关键字风格
	commentstyle        =   \rmfamily\itshape,  % 注释的风格,斜体
	stringstyle         =   \ttfamily,  % 字符串风格
	flexiblecolumns,                % 别问为什么,加上这个
	numbers             =   left,   % 行号的位置在左边
	showspaces          =   false,  % 是否显示空格,显示了有点乱,所以不现实了
	numberstyle         =   \zihao{-5}\ttfamily,    % 行号的样式,小五号,tt等宽字体
	showstringspaces    =   false,
	captionpos          =   t,      % 这段代码的名字所呈现的位置,t指的是top上面
	frame               =   lrtb,   % 显示边框
}

\lstdefinestyle{PythonStyle}{
	language        =   Python, % 语言选Python
	basicstyle      =   \zihao{-5}\ttfamily,
	numberstyle     =   \zihao{-5}\ttfamily,
	keywordstyle    =   \color{blue},
	keywordstyle    =   [2] \color{teal},
	stringstyle     =   \color{magenta},
	commentstyle    =   \color{red}\ttfamily,
	breaklines      =   true,   % 自动换行,建议不要写太长的行
	columns         =   fixed,  % 如果不加这一句,字间距就不固定,很丑,必须加
	basewidth       =   0.5em,
}

\algnewcommand\algorithmicinput{\textbf{Input:}}
\algnewcommand\algorithmicoutput{\textbf{Output:}}
\algnewcommand\Input{\item[\algorithmicinput]}%
\algnewcommand\Output{\item[\algorithmicoutput]}%

\usetikzlibrary{positioning, shapes.geometric}

\begin{document}
	
%%%%%%%%%%%%%%%%%%%%%%%%%%%%%%%%%%%%%%%%%%%%
\makeatletter % change default title style
\renewcommand*\maketitle{%
	\begin{center} 
		\bfseries  % title 
		{\LARGE \@title \par}  % LARGE typesetting
		\vskip 1em  %  margin 1em
		{\global\let\author\@empty}  % no author information
		{\global\let\date\@empty}  % no date
		\thispagestyle{empty}   %  empty page style
	\end{center}%
	\setcounter{footnote}{0}%
}
\makeatother
%%%%%%%%%%%%%%%%%%%%%%%%%%%%%%%%%%%%%%%%%%%%
	
	
\thispagestyle{empty}

\vspace*{1cm}

\begin{figure}[h]
	\centering
	\includegraphics[width=4.0cm]{logo.png}
\end{figure}

\vspace*{1cm}

\begin{center}
	\Huge{\textbf{\PaperType}}
	
	\Large{\PaperTitle}
\end{center}

\vspace*{1cm}

\begin{table}[h]
	\centering	
	\begin{Large}
		\renewcommand{\arraystretch}{1.5}
		\begin{tabular}{p{3cm} p{5cm}<{\centering}}
			姓\qquad 名 & \StudentName  \\
			\hline
			学\qquad号 & \StudentNumber \\
			\hline
			日\qquad期 & \Date  \\
			\hline
			学\qquad院 & \College  \\
			\hline
			课程名称 & \CourseName  \\
			\hline
		\end{tabular}
	\end{Large}
\end{table}

\newpage

\title{
	\Large{\textcolor{black}{\PaperTitle}}
}
	
	
\maketitle
	
\tableofcontents
 
\newpage
\setcounter{page}{1}

\begin{spacing}{1.2}

\section{问题描述}

\section{实现思路}

要求给定两个程序,判断它们的相似性。显然,\emph{程序的相似性和代码字符串的相似性无关,而与实际执行逻辑的相似性有关},例如\ref{fig:dupcode}中有两段代码本身不尽相同的代码,但是执行逻辑完全一致。那么最准确的方式是进行DFA(Data Flow Analysis)和CFA(Control Flow Analysis),对于相似的程序它们应当能相当准确地反映出相似度。这正是现代IDE对重复代码给出修改建议的方式。但这种方式实现相当复杂,本程序通过对问题进行简化有效地实现了\emph{\textbf{基于语义的}}代码相似性判断。

\begin{figure}[htbp]
\caption{逻辑相同但代码本身差异较大的代码}
\label{fig:dupcode}
\begin{minipage}{0.48\textwidth}
\begin{lstlisting}[language=Python,numbers=left,style=PythonStyle,label={code:dupexample1}]
int main()
{
	bool a = true;
	if (a)
	{
		printf("helloworld");
	}
	else
	{
		printf("worldhello");
	}
	return 0;
}
\end{lstlisting}
\end{minipage}
\begin{minipage}{0.48\textwidth}
\begin{lstlisting}[language=Python,numbers=left,style=PythonStyle,label={code:dupexample2}]
int main()
{
	int the_flag = 1;
	if (the_flag)
	{
		puts("helloworld");
	}
	else
	{
		puts("worldhello");
	}
	return 0;
}
\end{lstlisting}
\end{minipage}
\end{figure}

\subsection{两点假设}
为了简化问题,首先进行以下两个假设

\begin{itemize}
	\item 程序的抽象语法树(AST, Abstract Syntax Tree)和实际执行逻辑高度相关
	\item 抽象语法树中的语句节点和表达式节点是所有节点中和实际执行逻辑最相关的两类节点
\end{itemize}

根据这两点假设,通过从程序编译时的抽象语法树的语句(Statement)节点和表达式(Expression)节点序列中寻找最长公共子序列可以有效地衡量程序的逻辑相似性。定义逻辑相似度$s$,其中$AST_j$是程序代码$j$的抽象语法树的语句节点和表达式节点序列

$$
s=\frac{|LCS_{AST_1,AST_2}|}{\mathop{\max}\{|AST_1|,|AST_2|\}}
$$

\subsection{算法}

根据以上分析,首先需要获得AST中Statement和Expression组成的序列。这里借助Clang编译器。通过观察可以发现输出结果中Statement和Expression都符合正则表达式\eqref{eqn:stmtexprreg}。

\begin{equation}
\label{eqn:stmtexprreg}
([a-zA-Z\backslash\_][0-9a-zA-Z\backslash\_]*Expr)|([a-zA-Z\backslash\_][0-9a-zA-Z\backslash\_]*Stmt)
\end{equation}

因此可以用正则表达式匹配输出来获得上述序列。

\begin{figure}[htbp]
	\centering
	\caption{查重流程图设计}
	
	\vspace{0.5cm}
	\begin{tikzpicture}[node distance=10pt]
		\node[draw, rounded corners]                        (start)   {开始};
		\node[draw, below=of start]                         (step 1)  {输入两个代码文件名};
		\node[draw, below=of step 1]                        (step 2)  {从Clang编译器获取AST信息};
		\node[draw,  below=of step 2]  (step 3) {用正则表达式匹配出Statement和Expression序列};
		\node[draw,below=of step 3]  (step 4) {对序列求LCS长度};
		\node[draw, below=of step 4] (step 5) {计算重复率 $s$};
		\node[draw, rounded corners, below=of step 5]  (end)     {结束};
		\draw[->] (start)  -- (step 1);
		\draw[->] (step 1) -- (step 2);
		\draw[->] (step 2) -- (step 3);
		\draw[->] (step 3) -- (step 4);
		\draw[->] (step 4) -- (step 5);
		\draw[->] (step 5) -- (end);
	\end{tikzpicture}
	\label{fig:dupcheckflow}
\end{figure}

\newpage

根据以上分析可以得到算法流程图~\ref{fig:dupcheckflow},然后可以实现核心部分如代码~\ref{code:dupcheck},完整代码见\nameref{sec:app_impl}节中的代码~\ref{code:dupcheckfull}。该实现依赖Clang编译器。

\begin{lstlisting}[language=Python,numbers=left,style=PythonStyle,caption=核心部分代码,label={code:dupcheck}]
def lcs(s1, s2):
    f = [[0] * len(s1) * 2] * len(s2) * 2

    for i in range(1, len(s1) + 1):
        for j in range(1, len(s2) + 1):
            if s1[i - 1] == s2[j - 1]:
                f[i][j] = 1 + f[i - 1][j - 1]
            else:
                f[i][j] = max(f[i - 1][j], f[i][j - 1])

    return f[len(s1)][len(s2)]


def duplication_check(file1: str, file2: str):
    ast1 = str(shell(['clang -cc1 -ast-dump {}'.format(file1)]))
    ast2 = str(shell(['clang -cc1 -ast-dump {}'.format(file2)]))

    elems1 = list([i[0] if i[0] != '' else i[1] for i in
                   re.findall(r'([a-zA-Z\_][0-9a-zA-Z\_]*Expr)|([a-zA-Z\_][0-9a-zA-Z\_]*Stmt)', ast1)])
    elems2 = list([i[0] if i[0] != '' else i[1] for i in
                   re.findall(r'([a-zA-Z\_][0-9a-zA-Z\_]*Expr)|([a-zA-Z\_][0-9a-zA-Z\_]*Stmt)', ast2)])

    print("Repeat Rate: {}%".format((lcs(elems1, elems2) / max(len(elems1), len(elems2))) * 100.0))
\end{lstlisting}

使用以上程序对图~\ref{fig:dupcode}中的代码进行查重,输出的重复率是$81.25\%$,可以看出比起直接比较代码文面,该算法\emph{有效地反映出了底层逻辑的相似性,排除了修改变量名等传统降重方法造成的干扰}。

\subsection{结果分析}

根据上述测试结果可以得出结论,通过AST得节点序列可以有效地刻画程序的相似度。然而这种方法的思路依然是使用更高层级的“形式相似性”近似“逻辑相似性”,它考虑了一定程度的语义信息。因此,可以使用更为高级的降重技巧规避,例如

\begin{itemize}
	\item 改变语句顺序
	\item 将递归结构改为非递归结构
	\item 将一部分代码移动到一个子过程中
\end{itemize}

因此这种方式相比于正规的程序静态分析手段而言还是有不足的。然而这种方法实现计算简单,计算量较少,在实际使用中有一定的优势。


\clearpage
\appendix

\section{附录: 代码实现}
\label{sec:app_impl}

\subsection{矩阵连乘}
\begin{lstlisting}[language=Python,numbers=left,style=PythonStyle,caption=矩阵连乘,label={code:implmat}]
p = list([5, 10, 3, 12, 5, 50, 6])
N = 6

m = list([list([0x7fffffff for i in range(0, N + 1)]) for j in range(0, N + 1)])
s = list([list([0x7fffffff for i in range(0, N + 1)]) for j in range(0, N + 1)])


def pretty_print(i, j):
    if i == j:
        print('A{}'.format(i), end=' ')
    else:
        print("(", end=' ')
        pretty_print(i, s[i][j])
        pretty_print(s[i][j] + 1, j)
        print(")", end=' ')


for i in range(1, N + 1):
    m[i][i] = 0

for i in range(N, 0, -1):
    for j in range(i, N + 1):
        if i == j:
            m[i][j] = 0
        else:
            for k in range(i, j):
                val = m[i][k] + m[k + 1][j] + p[i - 1] * p[j] * p[k]
                if m[i][j] > val:
                    m[i][j] = val
                    s[i][j] = k

for i in range(1, N + 1):
    for j in range(1, N + 1):
        print("inf" if m[i][j] == 0x7fffffff else m[i][j], end=' ' if j != N else '\n')

for i in range(1, N + 1):
    for j in range(1, N + 1):
        print("None" if s[i][j] == 0x7fffffff else s[i][j], end=' ' if j != N else '\n')

pretty_print(1, N)

\end{lstlisting}

\clearpage

\subsection{查重程序}

注意: \emph{\textbf{该实现依赖Clang编译器}}

\begin{lstlisting}[language=Python,numbers=left,style=PythonStyle,caption=查重程序,label={code:dupcheckfull}]
import subprocess
from typing import Final
import re
import argparse


def shell(command):
    try:
        return subprocess.check_output(command, shell=True, stderr=subprocess.STDOUT).stdout
    except subprocess.CalledProcessError as exc:
        return exc.output


def lcs(s1, s2):
    f = [[0] * len(s1) * 2] * len(s2) * 2

    for i in range(1, len(s1) + 1):
        for j in range(1, len(s2) + 1):
            if s1[i - 1] == s2[j - 1]:
                f[i][j] = 1 + f[i - 1][j - 1]
            else:
                f[i][j] = max(f[i - 1][j], f[i][j - 1])

    return f[len(s1)][len(s2)]


def duplication_check(file1: str, file2: str):
    ast1 = str(shell(['clang -cc1 -ast-dump {}'.format(file1)]))
    ast2 = str(shell(['clang -cc1 -ast-dump {}'.format(file2)]))

    elems1 = list([i[0] if i[0] != '' else i[1] for i in
                   re.findall(r'([a-zA-Z\_][0-9a-zA-Z\_]*Expr)|([a-zA-Z\_][0-9a-zA-Z\_]*Stmt)', ast1)])
    elems2 = list([i[0] if i[0] != '' else i[1] for i in
                   re.findall(r'([a-zA-Z\_][0-9a-zA-Z\_]*Expr)|([a-zA-Z\_][0-9a-zA-Z\_]*Stmt)', ast2)])

    print("Repeat Rate: {}%".format((lcs(elems1, elems2) / max(len(elems1), len(elems2))) * 100.0))


if __name__ == "__main__":
    parser = argparse.ArgumentParser(prog="dupcheck",
                                     usage='%(prog)s [options] file1 file2',
                                     description="Duplication checker")
    parser.add_argument("file1")
    parser.add_argument("file2")

    args = parser.parse_args()

    duplication_check(str(args.file1), str(args.file2))

\end{lstlisting}

\end{spacing}
\end{document}
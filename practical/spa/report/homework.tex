\documentclass[a4paper,twoside]{article}
\usepackage{blindtext}  
\usepackage{geometry}

% Chinese support
\usepackage[UTF8, scheme = plain]{ctex}

% Page margin layout
\geometry{left=2.3cm,right=2cm,top=2.5cm,bottom=2.0cm}


\usepackage{listings}
\usepackage{xcolor}
\usepackage{geometry}
\usepackage{amsmath}
\usepackage{float}
\usepackage{hyperref}

\usepackage{graphics}
\usepackage{graphicx}
\usepackage{subfigure}
\usepackage{epsfig}
\usepackage{float}

\usepackage{algorithm}
\usepackage[noend]{algpseudocode}

\usepackage{booktabs}
\usepackage{threeparttable}
\usepackage{longtable}
\usepackage{listings}

% cite package, to clean up citations in the main text. Do not remove.
\usepackage{cite}

\usepackage{color,xcolor}

%% The amssymb package provides various useful mathematical symbols
\usepackage{amssymb}
%% The amsthm package provides extended theorem environments
\usepackage{amsthm}
\usepackage{amsfonts}
\usepackage{enumerate}
\usepackage{enumitem}
\usepackage{listings}

\usepackage{indentfirst}
\setlength{\parindent}{2em} % Make two letter space in the first paragraph
\usepackage{setspace}
\linespread{1.5} % Line spacing setting
\usepackage{siunitx}
\setlength{\parskip}{0.5em} % Paragraph spacing setting

\renewcommand{\figurename}{图}
\renewcommand{\lstlistingname}{代码} 
\renewcommand{\tablename}{表格}
\renewcommand{\contentsname}{目录}
\floatname{algorithm}{算法}

\graphicspath{ {images/} }

%%%%%%%%%%%%%
\newcommand{\StudentNumber}{22920202204622}  % Fill your student number here
\newcommand{\StudentName}{熊恪峥}  % Replace your name here
\newcommand{\PaperTitle}{实验报告(四)实现SPA算法}  % Change your paper title here
\newcommand{\PaperType}{算法设计与分析} % Replace the type of your report here
\newcommand{\Date}{2022年6月1日}
\newcommand{\College}{信息学院}
\newcommand{\CourseName}{算法设计与分析}
%%%%%%%%%%%%%

%% Page header and footer setting
\usepackage{fancyhdr}
\usepackage{lastpage}
\pagestyle{fancy}
\fancyhf{}
% This requires the document to be twoside
\fancyhead[LO]{\texttt{\StudentName }}
\fancyhead[LE]{\texttt{\StudentNumber}}
\fancyhead[C]{\texttt{\PaperTitle }}
\fancyhead[R]{\texttt{第{\thepage}页,共\pageref*{LastPage}页}}


\title{\PaperTitle}
\author{\StudentName}
\date{\Date}

\lstset{
	basicstyle          =   \sffamily,          % 基本代码风格
	keywordstyle        =   \bfseries,          % 关键字风格
	commentstyle        =   \rmfamily\itshape,  % 注释的风格,斜体
	stringstyle         =   \ttfamily,  % 字符串风格
	flexiblecolumns,                % 别问为什么,加上这个
	numbers             =   left,   % 行号的位置在左边
	showspaces          =   false,  % 是否显示空格,显示了有点乱,所以不现实了
	numberstyle         =   \zihao{-5}\ttfamily,    % 行号的样式,小五号,tt等宽字体
	showstringspaces    =   false,
	captionpos          =   t,      % 这段代码的名字所呈现的位置,t指的是top上面
	frame               =   lrtb,   % 显示边框
}

\lstdefinestyle{PythonStyle}{
	language        =   Python, % 语言选Python
	basicstyle      =   \zihao{-5}\ttfamily,
	numberstyle     =   \zihao{-5}\ttfamily,
	keywordstyle    =   \color{blue},
	keywordstyle    =   [2] \color{teal},
	stringstyle     =   \color{magenta},
	commentstyle    =   \color{red}\ttfamily,
	breaklines      =   true,   % 自动换行,建议不要写太长的行
	columns         =   fixed,  % 如果不加这一句,字间距就不固定,很丑,必须加
	basewidth       =   0.5em,
}

\begin{document}
	
%%%%%%%%%%%%%%%%%%%%%%%%%%%%%%%%%%%%%%%%%%%%
\makeatletter % change default title style
\renewcommand*\maketitle{%
	\begin{center} 
		\bfseries  % title 
		{\LARGE \@title \par}  % LARGE typesetting
		\vskip 1em  %  margin 1em
		{\global\let\author\@empty}  % no author information
		{\global\let\date\@empty}  % no date
		\thispagestyle{empty}   %  empty page style
	\end{center}%
	\setcounter{footnote}{0}%
}
\makeatother
%%%%%%%%%%%%%%%%%%%%%%%%%%%%%%%%%%%%%%%%%%%%
	
	
\thispagestyle{empty}

\vspace*{1cm}

\begin{figure}[h]
	\centering
	\includegraphics[width=4.0cm]{logo.png}
\end{figure}

\vspace*{1cm}

\begin{center}
	\Huge{\textbf{\PaperType}}
	
	\Large{\PaperTitle}
\end{center}

\vspace*{1cm}

\begin{table}[h]
	\centering	
	\begin{Large}
		\renewcommand{\arraystretch}{1.5}
		\begin{tabular}{p{3cm} p{5cm}<{\centering}}
			姓\qquad 名 & \StudentName  \\
			\hline
			学\qquad号 & \StudentNumber \\
			\hline
			日\qquad期 & \Date  \\
			\hline
			学\qquad院 & \College  \\
			\hline
			课程名称 & \CourseName  \\
			\hline
		\end{tabular}
	\end{Large}
\end{table}

\newpage

\title{
	\Large{\textcolor{black}{\PaperTitle}}
}
	
	
\maketitle
	
\tableofcontents
 
\newpage
\begin{spacing}{1.2}
%%%%%%%%%%%%%%%%%

\section{问题描述}

实现最短路增广算法解决最大流问题。

\section{实现思路}

Dinic算法是一种强多项式时间复杂度的算法,用来计算最大流。它的时间复杂度是$\mathcal{O}(VE^2)$。
这种算法在1970年由以色列计算机科学家Yefim A. Dinitzf提出,它优化了时间复杂度为$\mathcal{O}(V^2E)$的
Edmond-Karp算法。它的关键在于根据层级图进行增广路的寻找,如算法~\ref{algo:dinic}。

\begin{algorithm}[htbp]
	\caption{Dinic算法}
	\label{algo:dinic}
	\begin{algorithmic}[1]
		\Procedure{Dinic}{$G$}
			\For{$each\ e \in E$}
				\State $f(e)=0$
			\EndFor
			\State Compute $G_L$ from $G_f$
			\While{there is a path to $t$ in $G_L$}
				\State find a blocking flow $f'\in G_L$
				\State augumenting flow $f$ by $f'$
				\State Compute $G_L$ from $G_f$
			\EndWhile
		\EndProcedure
	\end{algorithmic}
\end{algorithm}	

\section{算法分析}

计算$G_L$需要进行广度优先搜索,它的时间复杂度是$\mathcal{O}(E)$,在$G_L$中寻找阻塞流需要$\mathcal{O}(VE)$
因此Dinic算法的时间复杂度是$\mathcal{O}(VE^2)$。

使用动态树的高级数据结构可以进一步将寻找阻塞流的时间复杂度缩减到$\mathcal{O}(\log VE)$,
时间复杂度将会降低到$\mathcal{O}(VE\log V)$。
\section{实现代码}

\begin{lstlisting}[language=Python,numbers=left,style=PythonStyle,caption=红包发放,label={code:redpacket}]
	
import random
from typing import Final, List

scheme_diff: List[int] = list({1.65, 1.67, 16.79, 1.77, 17.79, 1.87, 18.79, 1.98,
	5.19, 0.65, 6.59, 6.65, 0.07, 0.87, 8.79, 8.87, 0.98, 9.89, 9.98})

m, n = input("m and n:").split()

n = int(n)
m = float(m) - n * 0.01

result = list([0.01 for i in range(0, n)])

for i in range(0, n):
	amount = scheme_diff[random.randrange(0, len(scheme_diff))]
	if m - amount == 0:
		result[i] += amount
		break
	elif m - amount < 0:
		result[i] += m
		break
	else:
		m -= amount
		result[i] += amount

result = list([round(r, 2) for r in result])

print(result)

\end{lstlisting}

\end{spacing}
\end{document}
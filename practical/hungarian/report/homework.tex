\documentclass[a4paper,twoside]{article}
\usepackage{blindtext}  
\usepackage{geometry}

% Chinese support
\usepackage[UTF8, scheme = plain]{ctex}

% Page margin layout
\geometry{left=2.3cm,right=2cm,top=2.5cm,bottom=2.0cm}


\usepackage{listings}
\usepackage{xcolor}
\usepackage{geometry}
\usepackage{amsmath}
\usepackage{float}
\usepackage{hyperref}

\usepackage{graphics}
\usepackage{graphicx}
\usepackage{subfigure}
\usepackage{epsfig}
\usepackage{float}

\usepackage{algorithm}
\usepackage[noend]{algpseudocode}

\usepackage{booktabs}
\usepackage{threeparttable}
\usepackage{longtable}
\usepackage{listings}

% cite package, to clean up citations in the main text. Do not remove.
\usepackage{cite}

\usepackage{color,xcolor}

%% The amssymb package provides various useful mathematical symbols
\usepackage{amssymb}
%% The amsthm package provides extended theorem environments
\usepackage{amsthm}
\usepackage{amsfonts}
\usepackage{enumerate}
\usepackage{enumitem}
\usepackage{listings}

\usepackage{indentfirst}
\setlength{\parindent}{2em} % Make two letter space in the first paragraph
\usepackage{setspace}
\linespread{1.5} % Line spacing setting
\usepackage{siunitx}
\setlength{\parskip}{0.5em} % Paragraph spacing setting

\renewcommand{\figurename}{图}
\renewcommand{\lstlistingname}{代码} 
\renewcommand{\tablename}{表格}
\renewcommand{\contentsname}{目录}
\floatname{algorithm}{算法}

\newtheorem{theorem}{定理}

\graphicspath{ {images/} }

%%%%%%%%%%%%%
\newcommand{\StudentNumber}{22920202204622}  % Fill your student number here
\newcommand{\StudentName}{熊恪峥}  % Replace your name here
\newcommand{\PaperTitle}{实验报告(三)实现匈牙利算法}  % Change your paper title here
\newcommand{\PaperType}{算法设计与分析} % Replace the type of your report here
\newcommand{\Date}{2022年6月1日}
\newcommand{\College}{信息学院}
\newcommand{\CourseName}{算法设计与分析}
%%%%%%%%%%%%%

%% Page header and footer setting
\usepackage{fancyhdr}
\usepackage{lastpage}
\pagestyle{fancy}
\fancyhf{}
% This requires the document to be twoside
\fancyhead[LO]{\texttt{\StudentName }}
\fancyhead[LE]{\texttt{\StudentNumber}}
\fancyhead[C]{\texttt{\PaperTitle }}
\fancyhead[R]{\texttt{第{\thepage}页,共\pageref*{LastPage}页}}


\title{\PaperTitle}
\author{\StudentName}
\date{\Date}

\lstset{
	basicstyle          =   \sffamily,          % 基本代码风格
	keywordstyle        =   \bfseries,          % 关键字风格
	commentstyle        =   \rmfamily\itshape,  % 注释的风格,斜体
	stringstyle         =   \ttfamily,  % 字符串风格
	flexiblecolumns,                % 别问为什么,加上这个
	numbers             =   left,   % 行号的位置在左边
	showspaces          =   false,  % 是否显示空格,显示了有点乱,所以不现实了
	numberstyle         =   \zihao{-5}\ttfamily,    % 行号的样式,小五号,tt等宽字体
	showstringspaces    =   false,
	captionpos          =   t,      % 这段代码的名字所呈现的位置,t指的是top上面
	frame               =   lrtb,   % 显示边框
}

\lstdefinestyle{PythonStyle}{
	language        =   Python, % 语言选Python
	basicstyle      =   \zihao{-5}\ttfamily,
	numberstyle     =   \zihao{-5}\ttfamily,
	keywordstyle    =   \color{blue},
	keywordstyle    =   [2] \color{teal},
	stringstyle     =   \color{magenta},
	commentstyle    =   \color{red}\ttfamily,
	breaklines      =   true,   % 自动换行,建议不要写太长的行
	columns         =   fixed,  % 如果不加这一句,字间距就不固定,很丑,必须加
	basewidth       =   0.5em,
}

\lstdefinestyle{CppStyle}{
	language        =   c++,
	basicstyle      =   \zihao{-5}\ttfamily,
	numberstyle     =   \zihao{-5}\ttfamily,
	keywordstyle    =   \color{blue},
	keywordstyle    =   [2] \color{teal},
	stringstyle     =   \color{magenta},
	commentstyle    =   \color{red}\ttfamily,
	breaklines      =   true,   % 自动换行,建议不要写太长的行
	columns         =   fixed,  % 如果不加这一句,字间距就不固定,很丑,必须加
	basewidth       =   0.5em,
}

\begin{document}
	
%%%%%%%%%%%%%%%%%%%%%%%%%%%%%%%%%%%%%%%%%%%%
\makeatletter % change default title style
\renewcommand*\maketitle{%
	\begin{center} 
		\bfseries  % title 
		{\LARGE \@title \par}  % LARGE typesetting
		\vskip 1em  %  margin 1em
		{\global\let\author\@empty}  % no author information
		{\global\let\date\@empty}  % no date
		\thispagestyle{empty}   %  empty page style
	\end{center}%
	\setcounter{footnote}{0}%
}
\makeatother
%%%%%%%%%%%%%%%%%%%%%%%%%%%%%%%%%%%%%%%%%%%%
	
	
\thispagestyle{empty}

\vspace*{1cm}

\begin{figure}[h]
	\centering
	\includegraphics[width=4.0cm]{logo.png}
\end{figure}

\vspace*{1cm}

\begin{center}
	\Huge{\textbf{\PaperType}}
	
	\Large{\PaperTitle}
\end{center}

\vspace*{1cm}

\begin{table}[h]
	\centering	
	\begin{Large}
		\renewcommand{\arraystretch}{1.5}
		\begin{tabular}{p{3cm} p{5cm}<{\centering}}
			姓\qquad 名 & \StudentName  \\
			\hline
			学\qquad号 & \StudentNumber \\
			\hline
			日\qquad期 & \Date  \\
			\hline
			学\qquad院 & \College  \\
			\hline
			课程名称 & \CourseName  \\
			\hline
		\end{tabular}
	\end{Large}
\end{table}

\newpage

\title{
	\Large{\textcolor{black}{\PaperTitle}}
}
	
	
\maketitle
	
\tableofcontents
 
\newpage
\begin{spacing}{1.2}
%%%%%%%%%%%%%%%%%

\section{问题描述}

实现匈牙利算法解决二分图匹配算法

\section{问题分析}

匈牙利算法由匈牙利数学家于1965年提出,以Hall定理为依据,通过寻找增广路
来寻找二分图最大匹配。

由König定理:
\begin{theorem}
	一个二分图中的最大匹配数等于这个图中的最小点覆盖数。
\end{theorem}
该方法还可以用来处理最小点覆盖数问题。也就是想找到最少的一些点,使得二分图
所有的边都至少有一个端点在这些点中。

匈牙利算法的执行过程是

\begin{itemize}
	\item 从点集$\mathcal{V}$中找到未匹配点
	\item 寻找增广路
	\item 如果找到,就记录匹配
	\item 如果没有找到,就从$\mathcal{V}$中找下一个未匹配的点
\end{itemize}
	
在这个过程中,匈牙利算法不断地寻找增广路,直到无法找到新的增广路。因此
可以找到最大匹配。

\appendix
\section{实现代码}

\begin{lstlisting}[language=c++,numbers=left,style=CppStyle,caption=红包发放,label={code:redpacket}]
#include<iostream>

using namespace std;

constexpr int NMAX = 1010;
constexpr int EMAX = 1000010;
int n = 0, m = 0, e = 0, ans = 0;

int edge[EMAX], head[NMAX], nxt[EMAX], tot = 0;
int dfn[NMAX << 1], match[NMAX], x = 0, y = 0, ti = 0;

void add(int x, int y)
{
	edge[++tot] = y;
	nxt[tot] = head[x];
	head[x] = tot;
}

int hungrain(int x, int ti)
{
	for (int i = head[x]; i; i = nxt[i])
	{
		int j = edge[i];
		if (dfn[j] != ti)
		{
			dfn[j] = ti;
			if (!match[j] || hungrain(match[j], ti))
			{
				match[j] = i;
				return 1;
			}
		}

	}
	return 0;
}

int main()
{
	cin >> n >> m >> e;
	for (int i = 1; i <= e; i++)
	{
		cin >> x >> y;
		if (x > n || y > m)
		{
			continue;
		}
		add(x, y);
	}

	for (int i = 1; i <= n; i++)
	{
		if (hungrain(i, ++ti))
		{
			ans++;
		}
	}

	cout << ans << endl;

	return 0;
}
\end{lstlisting}

\end{spacing}
\end{document}
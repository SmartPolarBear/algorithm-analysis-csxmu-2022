\documentclass[a4paper,twoside]{article}
\usepackage{blindtext}  
\usepackage{geometry}

% Chinese support
\usepackage[UTF8, scheme = plain]{ctex}

% Page margin layout
\geometry{left=2.3cm,right=2cm,top=2.5cm,bottom=2.0cm}


\usepackage{listings}
\usepackage{xcolor}
\usepackage{geometry}
\usepackage{amsmath}
\usepackage{float}
\usepackage{hyperref}

\usepackage{graphics}
\usepackage{graphicx}
\usepackage{subfigure}
\usepackage{epsfig}
\usepackage{float}

\usepackage{algorithm}
\usepackage[noend]{algpseudocode}

\usepackage{booktabs}
\usepackage{threeparttable}
\usepackage{longtable}
\usepackage{listings}

% cite package, to clean up citations in the main text. Do not remove.
\usepackage{cite}

\usepackage{color,xcolor}

%% The amssymb package provides various useful mathematical symbols
\usepackage{amssymb}
%% The amsthm package provides extended theorem environments
\usepackage{amsthm}
\usepackage{amsfonts}
\usepackage{enumerate}
\usepackage{enumitem}
\usepackage{listings}

\usepackage{indentfirst}
\setlength{\parindent}{2em} % Make two letter space in the first paragraph
\usepackage{setspace}
\linespread{1.5} % Line spacing setting
\usepackage{siunitx}
\setlength{\parskip}{0.5em} % Paragraph spacing setting

% \usepackage[contents =22920202204622, scale = 10, color = black, angle = 50, opacity = .10]{background}

\renewcommand{\figurename}{图}
\renewcommand{\lstlistingname}{代码} 
\renewcommand{\tablename}{表格}
\renewcommand{\contentsname}{目录}
\floatname{algorithm}{算法}

\graphicspath{ {images/} }

%%%%%%%%%%%%%
\newcommand{\StudentNumber}{22920202204622}  % Fill your student number here
\newcommand{\StudentName}{熊恪峥}  % Replace your name here
\newcommand{\PaperTitle}{作业(二)}  % Change your paper title here
\newcommand{\PaperType}{算法设计与分析} % Replace the type of your report here
\newcommand{\Date}{2022年2月28日}
\newcommand{\College}{信息学院}
\newcommand{\CourseName}{算法设计与分析}
%%%%%%%%%%%%%

%% Page header and footer setting
\usepackage{fancyhdr}
\usepackage{lastpage}
\pagestyle{fancy}
\fancyhf{}
% This requires the document to be twoside
\fancyhead[LO]{\texttt{\StudentName }}
\fancyhead[LE]{\texttt{\StudentNumber}}
\fancyhead[C]{\texttt{\PaperTitle }}
\fancyhead[R]{\texttt{第{\thepage}页,共\pageref*{LastPage}页}}


\title{\PaperTitle}
\author{\StudentName}
\date{\Date}

\lstset{
	basicstyle          =   \sffamily,          % 基本代码风格
	keywordstyle        =   \bfseries,          % 关键字风格
	commentstyle        =   \rmfamily\itshape,  % 注释的风格,斜体
	stringstyle         =   \ttfamily,  % 字符串风格
	flexiblecolumns,                % 别问为什么,加上这个
	numbers             =   left,   % 行号的位置在左边
	showspaces          =   false,  % 是否显示空格,显示了有点乱,所以不现实了
	numberstyle         =   \zihao{-5}\ttfamily,    % 行号的样式,小五号,tt等宽字体
	showstringspaces    =   false,
	captionpos          =   t,      % 这段代码的名字所呈现的位置,t指的是top上面
	frame               =   lrtb,   % 显示边框
}

\lstdefinestyle{PythonStyle}{
	language        =   Python, % 语言选Python
	basicstyle      =   \zihao{-5}\ttfamily,
	numberstyle     =   \zihao{-5}\ttfamily,
	keywordstyle    =   \color{blue},
	keywordstyle    =   [2] \color{teal},
	stringstyle     =   \color{magenta},
	commentstyle    =   \color{red}\ttfamily,
	breaklines      =   true,   % 自动换行,建议不要写太长的行
	columns         =   fixed,  % 如果不加这一句,字间距就不固定,很丑,必须加
	basewidth       =   0.5em,
}

\begin{document}
	
%%%%%%%%%%%%%%%%%%%%%%%%%%%%%%%%%%%%%%%%%%%%
\makeatletter % change default title style
\renewcommand*\maketitle{%
	\begin{center} 
		\bfseries  % title 
		{\LARGE \@title \par}  % LARGE typesetting
		\vskip 1em  %  margin 1em
		{\global\let\author\@empty}  % no author information
		{\global\let\date\@empty}  % no date
		\thispagestyle{empty}   %  empty page style
	\end{center}%
	\setcounter{footnote}{0}%
}
\makeatother
%%%%%%%%%%%%%%%%%%%%%%%%%%%%%%%%%%%%%%%%%%%%
	
	
\thispagestyle{empty}

\vspace*{1cm}

\begin{figure}[h]
	\centering
	\includegraphics[width=4.0cm]{logo.png}
\end{figure}

\vspace*{1cm}

\begin{center}
	\Huge{\textbf{\PaperType}}
	
	\Large{\PaperTitle}
\end{center}

\vspace*{1cm}

\begin{table}[h]
	\centering	
	\begin{Large}
		\renewcommand{\arraystretch}{1.5}
		\begin{tabular}{p{3cm} p{5cm}<{\centering}}
			姓\qquad 名 & \StudentName  \\
			\hline
			学\qquad号 & \StudentNumber \\
			\hline
			日\qquad期 & \Date  \\
			\hline
			学\qquad院 & \College  \\
			\hline
			课程名称 & \CourseName  \\
			\hline
		\end{tabular}
	\end{Large}
\end{table}

\newpage

\title{
	\Large{\textcolor{black}{\PaperTitle}}
}
	
	
\maketitle
	
\tableofcontents
 
\newpage
\begin{spacing}{1.2}
	

\section{题2.2}
由$\max$的定义

\begin{align*}
	f(x) \le \max (f(x),g(x)) \\
	g(x) \le \max (f(x),g(x)) 
\end{align*}

则

$$
f(x)+g(x) \le 2 \max (f(x),g(x))
$$

则有

$$
\max (f(x),g(x)) \ge \frac{f(x)+g(x)}{2}
$$

即

\begin{equation}
	\max (f(x),g(x)) = \Omega (f(x),g(x)) \label{p22theta}
\end{equation}

由非负性

$$
\max (f(x),g(x)) \le f(x)+g(x) 
$$

即

\begin{equation}
	\max (f(x),g(x)) = O (f(x)+g(x)) \label{eqn:p22o}
\end{equation}

由\eqref{p22theta}和\eqref{eqn:p22o}可得


$$
\max (f(x),g(x)) = \Theta (f(x)+g(x)) \label{eqn:p22o}
$$

\section{题2.9}

\textbf{必要性:} 
由$f(n)=\Theta(g(n))$得
$$
\lim_{n \to \infty}\frac{f(n)}{g(n)}=k, 0<k<\infty
$$

则由

$$
\lim_{n \to \infty}\frac{f(n)}{g(n)} \ne 0
$$

得$f(n)=\Omega(g(n))$,由

$$
\lim_{n \to \infty}\frac{f(n)}{g(n)} \ne \infty
$$

得$f(n)=O(g(n))$

\textbf{充分性:}
由$f(n)=O(g(n))$且$f(n)=\Omega(g(n))$得

$$
\lim_{n \to \infty}\frac{f(n)}{g(n)} \ne \infty
$$

且

$$
\lim_{n \to \infty}\frac{f(n)}{g(n)} \ne 0
$$

则$\exists k>0$且$k<\infty$使

$$
\lim_{n \to \infty}\frac{f(n)}{g(n)} = k
$$

则$f(n)=\Theta(g(n))$

\section{题3.1}

\textbf{正好雇佣一次},则第一位面试的必须是最好的面试者。
则
$$
P(\mbox{排在第$1$位的面试者是最好的面试者})=\frac{1}{n}
$$

\textbf{正好雇佣两次},则已知排在第一位的人一定会被雇佣,最好的面试者一定会被雇佣。因此,第一位不能是最好的的面试者,否则只能被雇佣一次。则设事件

$$
E_i: \mbox{第一个来面试的人的排名是$i$}
$$

其中$i$满足

$$
i \le n-1
$$

则

$$
P(E_i)=\frac{1}{n}
$$

若第一个人的排名是$i$,只雇佣两个人要求第$2, 3 \dots ,j-1$个面试者排名都不如第一个面试者,即最好的人必须在排名是$i+1, i+2 \dots n-1, n$的人中第一个面试。设最好的人面试的次序是$j$,则事件

$$
F: \mbox{第$2, 3 \dots ,j-1$个面试者排名都不如第一个面试者}
$$

则

\begin{align*}
	\overbrace{i+1, i+2, \dots, n-1, n}^{\mbox{共$n-i$个人}} \\
	\Rightarrow P(F|E_i) = \frac{1}{n-i} 
\end{align*}

注意到$E_1, \dots, E_{n-1}$是独立事件,则由全概率公式

\begin{align*}
	P(\mbox{恰好有两个人被雇佣}) &= \sum_{i=1}^{n-1}{P(F|E_i)\times P(E_i)} \\
	&= \sum_{i=1}^{n-1}\frac{1}{n} \cdot \frac{1}{n-i} \\
	&= \frac{1}{n}\sum_{i=1}^{n-1}\frac{1}{i} \\
	&= O(\frac{\log n}{n})
\end{align*}

\textbf{正好雇佣$n$次},则所有候选人按排名单调递增进行面试。总共有$n!$种排列,则概率为

$$
P(\mbox{恰好有$n$个人被雇佣})=\frac{1}{n!}
$$

\section{题3.2}

$FindMax$算法如算法\ref{alg:findmax}

\begin{algorithm}
	\caption{查找最大值,返回下标}
	\label{alg:findmax}
	\begin{algorithmic}[1]
		\Procedure{FindMax}{$A$}     
		\State $max \gets 1$
		\For{$j \gets 2$  to $n$}
		\If{$A[j]>A[max]$}
		\State $max \gets j$
		\EndIf
		\EndFor
		\Return max
		\EndProcedure
		
	\end{algorithmic}
\end{algorithm}

则$max$在位置$k$被赋予最大值时的概率为
$$
P=\frac{1}{n}
$$

如果$max=A[k]$则第$3$行的比较次数为$n-1$次,则平均的比较次数为

\begin{align*}
	T(n) &= \sum_{k=2}^{n}\frac{1}{n} \cdot k \\
	&= \frac{1}{n} \cdot \frac{(2+n)(n-1)}{2} \\
	&= \Theta(n)
\end{align*}

\section{题3.4}

设运算$i$的开销为$c_i$

$$ c_i=\left\{
\begin{aligned}
	i & \  \mbox{$i$为$2$的整数幂} \\
	1 & \  \mbox{其它} \\
\end{aligned}
\right.
$$

则总开销为

\begin{align*}
	C &= \sum_{i=1}^{n}c_i \\
	&= \sum_{i=1}^{\lfloor\ log_2 n \rfloor}2^i+(n-\lfloor\ log_2 n \rfloor) \\
	&=  2(n-1) + (n-\lfloor\ log_2 n \rfloor)
\end{align*}

则平均一次运算的开销为

$$
\overline{c_i}=\frac{C}{n} = 3+\frac{1}{n}+\frac{\lfloor\ log_2 n \rfloor}{n} 
$$

当$n$较大时

\begin{align*}
	\overline{c_i}=\lim_{n \to \infty}\frac{C}{n} &= 3+\lim_{n \to \infty}(\frac{1}{n}+\frac{\lfloor\ log_2 n \rfloor}{n}) \\
	&=3
\end{align*}

\section{题3.6}

已知实际代价$c_i$为

$$ c_i=\left\{
\begin{aligned}
	i & \  \mbox{$i$为$2$的整数幂} \\
	1 & \  \mbox{其它} \\
\end{aligned}
\right.
$$

对每一个操作收费$\$3$($\hat{c_i}=3$),并且
\begin{itemize}
	\item 当$i$为2的整数幂,使用存款支付$i$
	\item 当$i$不是2的整数幂,支付费用$\$1$,并且增减存款$\$2$
\end{itemize}

可以得到表\ref{tbl:p36costs}

\begin{table}[h]
	\renewcommand{\arraystretch}{1.2}
	\caption{每次操作的代价与存款}
	\label{tbl:p36costs}
	\begin{center}
		{\footnotesize
			\begin{tabular}{|c|c|c|}
				\hline
				{$i$} & {代价} & {存款} \\
				\hline
				$1$ & 1 & 2 \\
				\hline
				$2$ & 2 & 3\\
				\hline
				$3$ & 1 & 5\\
				\hline
				$4$ & 4 & 4\\
				\hline
				$5$ & 1 & 6\\
				\hline
				$\dots$ & $\dots$ & $\dots$ \\
				\hline
		\end{tabular}}
	\end{center}
\end{table}

由表\ref{tbl:p36costs}可知$\hat{c_i}=3$满足

$$
C=3n=\sum_{i=0}^{n}\hat{c_i} \ge \sum_{i=0}^{n}{c_i}
$$

则平均一次的开销为

$$
 \overline{c_i}=\frac{C}{n}=3
$$

\section{题3.8}

已知实际代价$c_i$为

$$ c_i=\left\{
\begin{aligned}
	i & \  \mbox{$i$为$2$的整数幂} \\
	1 & \  \mbox{其它} \\
\end{aligned}
\right.
$$

根据势能法,有

\begin{align*}
	\hat{c_i}&=c_i+D_i-D_{i-1}\\
	&=\left\{
	\begin{aligned}
		i+D_i-D_{i-1} & \  \mbox{$i$为$2$的整数幂} \\
		1+D_i-D_{i-1} & \  \mbox{$i$不是$2$的整数幂} \\
	\end{aligned}
	\right.
\end{align*}


令$i=2^j+k$,定义势函数$D_i=2k$

$$
D_i-D_{i-1}=\left\{
\begin{aligned}
	0-2(2^j-2^{j-1}-1)&=-i+2 &  \mbox{$i$为$2$的整数幂} \\
	2k-2(k-1)&=2  &  \mbox{$i$不是$2$的整数幂} \\
\end{aligned}
\right.
$$

则


\begin{align*}
	\hat{c_i}&=c_i+D_i-D_{i-1}\\
	&=\left\{
	\begin{aligned}
		i-i+2 &= 2 & \  \mbox{$i$为$2$的整数幂} \\
		1+2 &= 3 & \  \mbox{$i$不是$2$的整数幂} \\
	\end{aligned}
	\right.
\end{align*}

求和可得

\begin{align*}
	C=\sum_{i=0}^{n}{\hat{c_i}}&=\sum_{j=0}^{\lfloor \log_2 n \rfloor}2+3(n-\lfloor \log_2 n \rfloor) \\
	&= 2{\lfloor \log_2 n \rfloor}+3(n-\lfloor \log_2 n \rfloor)
\end{align*}

平均一次的代价


\begin{align*}
 \overline{c_i}=\frac{C}{n}&= \frac{2{\lfloor \log_2 n \rfloor}+3(n-\lfloor \log_2 n \rfloor)}{n} \\
 & \le \lim_{n \to \infty}{3+\frac{2{\lfloor \log_2 n \rfloor}}{n}+\frac{\lfloor \log_2 n \rfloor}{n}} \\
 &= 3+0+0=3
\end{align*}

\end{spacing}
\end{document}
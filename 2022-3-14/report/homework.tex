\documentclass[a4paper,twoside]{article}
\usepackage{blindtext}  
\usepackage{geometry}

% Chinese support
\usepackage[UTF8, scheme = plain]{ctex}

% Page margin layout
\geometry{left=2.3cm,right=2cm,top=2.5cm,bottom=2.0cm}


\usepackage{listings}
\usepackage{xcolor}
\usepackage{geometry}
\usepackage{amsmath}
\usepackage{float}
\usepackage{hyperref}

\usepackage{graphics}
\usepackage{graphicx}
\usepackage{subfigure}
\usepackage{epsfig}
\usepackage{float}

\usepackage{algorithm}
\usepackage[noend]{algpseudocode}

\usepackage{booktabs}
\usepackage{threeparttable}
\usepackage{longtable}
\usepackage{listings}
\usepackage{tikz}

% cite package, to clean up citations in the main text. Do not remove.
\usepackage{cite}

\usepackage{color,xcolor}

%% The amssymb package provides various useful mathematical symbols
\usepackage{amssymb}
%% The amsthm package provides extended theorem environments
\usepackage{amsthm}
\usepackage{amsfonts}
\usepackage{enumerate}
\usepackage{enumitem}
\usepackage{listings}

\usepackage{indentfirst}
\setlength{\parindent}{2em} % Make two letter space in the first paragraph
\usepackage{setspace}
\linespread{1.5} % Line spacing setting
\usepackage{siunitx}
\setlength{\parskip}{0.5em} % Paragraph spacing setting

% \usepackage[contents =22920202204622, scale = 10, color = black, angle = 50, opacity = .10]{background}

\renewcommand{\figurename}{图}
\renewcommand{\lstlistingname}{代码} 
\renewcommand{\tablename}{表格}
\renewcommand{\contentsname}{目录}
\floatname{algorithm}{算法}

\graphicspath{ {images/} }

%%%%%%%%%%%%%
\newcommand{\StudentNumber}{22920202204622}  % Fill your student number here
\newcommand{\StudentName}{熊恪峥}  % Replace your name here
\newcommand{\PaperTitle}{作业(四)}  % Change your paper title here
\newcommand{\PaperType}{算法设计与分析} % Replace the type of your report here
\newcommand{\Date}{2022年3月14日}
\newcommand{\College}{信息学院}
\newcommand{\CourseName}{算法设计与分析}
%%%%%%%%%%%%%

%% Page header and footer setting
\usepackage{fancyhdr}
\usepackage{lastpage}
\pagestyle{fancy}
\fancyhf{}
% This requires the document to be twoside
\fancyhead[LO]{\texttt{\StudentName }}
\fancyhead[LE]{\texttt{\StudentNumber}}
\fancyhead[C]{\texttt{\PaperTitle }}
\fancyhead[R]{\texttt{第{\thepage}页,共\pageref*{LastPage}页}}


\title{\PaperTitle}
\author{\StudentName}
\date{\Date}

\lstset{
	basicstyle          =   \sffamily,          % 基本代码风格
	keywordstyle        =   \bfseries,          % 关键字风格
	commentstyle        =   \rmfamily\itshape,  % 注释的风格,斜体
	stringstyle         =   \ttfamily,  % 字符串风格
	flexiblecolumns,                % 别问为什么,加上这个
	numbers             =   left,   % 行号的位置在左边
	showspaces          =   false,  % 是否显示空格,显示了有点乱,所以不现实了
	numberstyle         =   \zihao{-5}\ttfamily,    % 行号的样式,小五号,tt等宽字体
	showstringspaces    =   false,
	captionpos          =   t,      % 这段代码的名字所呈现的位置,t指的是top上面
	frame               =   lrtb,   % 显示边框
}

\lstdefinestyle{PythonStyle}{
	language        =   Python, % 语言选Python
	basicstyle      =   \zihao{-5}\ttfamily,
	numberstyle     =   \zihao{-5}\ttfamily,
	keywordstyle    =   \color{blue},
	keywordstyle    =   [2] \color{teal},
	stringstyle     =   \color{magenta},
	commentstyle    =   \color{red}\ttfamily,
	breaklines      =   true,   % 自动换行,建议不要写太长的行
	columns         =   fixed,  % 如果不加这一句,字间距就不固定,很丑,必须加
	basewidth       =   0.5em,
}

\algnewcommand\algorithmicinput{\textbf{Input:}}
\algnewcommand\algorithmicoutput{\textbf{Output:}}
\algnewcommand\Input{\item[\algorithmicinput]}%
\algnewcommand\Output{\item[\algorithmicoutput]}%

\begin{document}
	
%%%%%%%%%%%%%%%%%%%%%%%%%%%%%%%%%%%%%%%%%%%%
\makeatletter % change default title style
\renewcommand*\maketitle{%
	\begin{center} 
		\bfseries  % title 
		{\LARGE \@title \par}  % LARGE typesetting
		\vskip 1em  %  margin 1em
		{\global\let\author\@empty}  % no author information
		{\global\let\date\@empty}  % no date
		\thispagestyle{empty}   %  empty page style
	\end{center}%
	\setcounter{footnote}{0}%
}
\makeatother
%%%%%%%%%%%%%%%%%%%%%%%%%%%%%%%%%%%%%%%%%%%%
	
	
\thispagestyle{empty}

\vspace*{1cm}

\begin{figure}[h]
	\centering
	\includegraphics[width=4.0cm]{logo.png}
\end{figure}

\vspace*{1cm}

\begin{center}
	\Huge{\textbf{\PaperType}}
	
	\Large{\PaperTitle}
\end{center}

\vspace*{1cm}

\begin{table}[h]
	\centering	
	\begin{Large}
		\renewcommand{\arraystretch}{1.5}
		\begin{tabular}{p{3cm} p{5cm}<{\centering}}
			姓\qquad 名 & \StudentName  \\
			\hline
			学\qquad号 & \StudentNumber \\
			\hline
			日\qquad期 & \Date  \\
			\hline
			学\qquad院 & \College  \\
			\hline
			课程名称 & \CourseName  \\
			\hline
		\end{tabular}
	\end{Large}
\end{table}

\newpage

\title{
	\Large{\textcolor{black}{\PaperTitle}}
}
	
	
\maketitle
	
\tableofcontents
 
\newpage
\setcounter{page}{1}

\begin{spacing}{1.2}

\section{题5.2}

实现非递归的归并排序进行一个自下而上的从2个元素的数组开始的合并。算法如算法\ref{algo:itermergesort},其中$merge$就是普通的二路归并实现。

\begin{algorithm}
	\caption{非递归归并排序}
	\label{algo:itermergesort}
	\begin{algorithmic}[1]
		\Procedure{MergeSort}{$A$,$n$}     
		\For{$size \gets 1$ to $n$ step $size$}
		\For{$left \gets 0$ to $n-1$ step $2size$}
		\State $mid \gets min(left+size-1,n-1)$
		\State $right \gets min(left+2size-1,n-1)$
		\State $merge(A,left,mid,right)$
		\EndFor
		\EndFor
		\EndProcedure
		
	\end{algorithmic}
\end{algorithm}	

通过画出区间划分的树状图,可以发现它与递归方法等效,并且有相同的时间复杂度,它的时间复杂度仍是$O(n\log n)$,这是基于比较的排序算法的时间复杂度下界,因此它是最有效的。

\section{题5.3}
如算法\ref{algo:recurbsearch},通过二分搜索的方法寻找$x$,在找到时返回下标,否则返回$-1$。

\begin{algorithm}
	\caption{二分搜索}
	\label{algo:recurbsearch}
	\begin{algorithmic}[1]
		\Procedure{Search}{$A$,$x$,$l$,$r$}     
		\State $mid \gets \frac{l+r}{2} $
		\If{$A[mid] == x$}
		\Return $mid$
		\EndIf
		
		\If{$l>=r$}
		\Return $-1$
		\ElsIf{$x>A[mid]$}
		\Return $Search(A,x,mid+1,r)$
		\ElsIf{$x<A[mid]$}
		\Return $Search(A,x,l,mid-1)$
		\EndIf
		
		\Return $-1$
		\EndProcedure
	\end{algorithmic}
\end{algorithm}	

\textbf{复杂度分析:}该算法有递推方程

$$
T(n)=T(\lfloor \frac{n}{2} \rfloor)+1
$$

可以证明
$$
T(n)=O(\log n)
$$

\textbf{证:}$T(n)<c\log n$

\textbf{当}$n=2$时$T(n)=1\le c\log2=c$

当

$$
c \ge 1
$$

\textbf{若}
$
T(\frac{n}{2}) \le c\log \frac{n}{2}
$
成立,则

\begin{align*}
	T(n)&=T(\frac{n}{2})+1 \\
	&\le c\log \frac{n}{2}+1 \\
	&= c\log n - c\log 2+1 \le c\log n
\end{align*}

当$
c \ge \frac{1}{\log 2}
$成立

\section{题5.8}

4路归并排序如算法\ref{algo:4waymergesort},把当前区间分成四份并对每一份分别进行归并排序,之后对左边的两个区间、右边的两个区间分别进行二路归并,此时得到了左、右两个有序子区间。再对这两个子区间进行再一次归并,得到整个有序序列。

\begin{algorithm}
	\caption{二分搜索}
	\label{algo:4waymergesort}
	\begin{algorithmic}[1]
		\Procedure{MergeSort}{$A$,$l$,$r$}     
		\State $mid \gets \frac{l+r}{2} $
		\State $lmid \gets \frac{l+mid}{2}$
		\State $rmid \gets \frac{mid+r}{2}$
		
		\State \Call{MergeSort}{A,l,lmid}
		\State \Call{MergeSort}{A,lmid,mid}
		\State \Call{MergeSort}{A,mid,rmid}
		\State \Call{MergeSort}{A,rmid,r}
		
		\State \Call{Merge}{A,l,lmid,mid}
		\State \Call{Merge}{A,mid,rmid,r}
		\State \Call{Merge}{A,l,mid,r}
		
		\EndProcedure
	\end{algorithmic}
\end{algorithm}	

\textbf{复杂度分析:}该算法满足递推方程

$$
T(n)=4T(\frac{n}{4})+2n
$$

因为每一个子问题的规模是$\frac{n}{4}$,合并子区间时先合并左右两边长为$\frac{n}{4}+\frac{n}{4}=\frac{n}{2}$的子区间,最后进行长度为$n$的子区间,由$Merge$操作的时间复杂度w为$O(n)$可知合并的代价是$2n$。则有

$$
T(n)=O(n\log n)
$$

\textbf{证:} $T(n)<cn\log n$

\textbf{当$n=4$}有$T(n)=8<c\cdot4\log 4$,当$c\ge 1$

\textbf{若}$T(\frac{n}{4})\le c\frac{n}{4}\log \frac{n}{4}$,则

\begin{align*}
	T(n)&=4T(\frac{n}{4})+2n \\
	&\le cn\log \frac{n}{4}+2n \\
	&=cn\log n+(2-2c\log 2)n \\
	&\le cn\log n
\end{align*}

当$c\ge \frac{1}{\log 2}$成立

\section{题5.10}

快速排序的$partition$操作能够做到使得小于枢轴($pivot$)的元素都处于枢轴左侧,大于枢轴的元素都处在枢轴右侧,根据这一性质,那么当一次$partition$完成时

\begin{itemize}
	\item 如果枢轴处在$k$处,那么它恰好是第$k$大
	\item 如果枢轴在$k$右侧,$k$比第$k$大更大,那么第$k$大的元素在枢轴左侧的子区间中,那么可以递归地继续寻找第$k$大
	\item 否则,第$k$大的元素在枢轴右侧的子区间中,递归地寻找第$k$大
\end{itemize}

根据以上思路,实现算法\ref{algo:select},其中$partition$就是快速排序算法中$partition$的实现,它的时间复杂度是$O(n)$

\begin{algorithm}
\caption{选择第$k$大元素}
\label{algo:select}
\begin{algorithmic}[1]
	\Procedure{Select}{$A$,$k$,$l$,$r$}     
	
	\State $pivot \gets $ \Call{partition}{A,l,r}
	
	\If{$pivot==k$}
	\State \Return $A[pivot]$
	\ElsIf{$pivot>k$}
	\State \Return \Call{Select}{A,k,l,pivot-1}
	\Else
	\State \Return \Call{Select}{A,k,pivot+1,r}
	\EndIf
	
	\State \Return $-1$
	\EndProcedure
\end{algorithmic}
\end{algorithm}

平均情况下,可以认为元素的大小随机分布,那么对于每一个元素大致有$\frac{n}{2}$个元素比它大或者比它小。考虑到$partition$的开销,那么有如下递推式

$$
T(n)=T(\frac{n}{2})+n=O(n)
$$

\textbf{证:}$T(n) \le cn$

\textbf{当}$n=1$,有$T(1)=1\le cn$,当$c\ge 1$

\textbf{若}$T(\frac{n}{2})<c\frac{n}{2}$成立,则

\begin{align*}
	T(n)&=T(\frac{n}{2})+n\\
	&\le c\frac{n}{2}+n =(1+\frac{n}{2})n \\
	&\le cn
\end{align*}

当$n\ge 2$时成立

\section{题5.18}


如图\ref{fig:chese}所示,在左上角放下一个三连格之后,可以将完整的棋盘分为两个完整的子棋盘,然后可以递归地继续解决这两个完整的子棋盘的铺设问题。

\begin{figure}[h]
	\centering
	\label{fig:chese}
	\includegraphics[width=0.5\linewidth]{cheseboard.png}
	\caption{放下一个三连格之后}
\end{figure}


这种方法是可行的,因为三连格的特点是

\begin{itemize}
	\item 短边一侧的长度是1
	\item 长边一侧的长度是3
\end{itemize}

对于一个$2i \times 3j$的平板而言,短边的长度可以密铺\textit{任意}一端($1$可以整除任何数),但是长边的长度\textit{需要对应$3j$的边长},这样才能不重叠地覆盖该棋盘。那么这个问题实际上相当于解决以下问题:

\begin{equation}
	\label{eqn:derived_problem}
	\mbox{\centerline{\textit{\textbf{使用三连格平铺一边长度是$3$的倍数的平板}}}}
\end{equation}

如图\ref{fig:chese},将一个三连格的长边对应平板长度是$3$的倍数的边,放置在左上角,这样一来由三连格的右侧切分平板,将平板(设为$3j \times h$)分为两部分:

\begin{itemize}
	\item $3 \times h$的子平板
	\item $3(j-1) \times h$的子平板
\end{itemize}

显然,用三连格平铺这两个部分都属于问题\eqref{eqn:derived_problem}的子问题,因为它们有一边长度是$3$的倍数。并且这两个子问题规模较原问题有所减少,因此可以用分治的方法解决,如算法\ref{algo:cover}。

\begin{algorithm}
	\caption{使用三连格平铺}
	\label{algo:cover}
	\begin{algorithmic}[1]
		\Input{$B$: 初始化为0的二维数组,表示平板, $w$: 宽度,为$3$的倍数, $h$: 高度}
		\Procedure{Cover}{$B[w][h]$,$w$,$h$,$x=0$,$y=0$,$r=0$}     
		
		\State $(B[x],B[x+1],B[x+2]) \gets (r,r,r)$
		
		\If{$w==3$ $\mathbf{and}$ $h==1$}
		\Return
		\EndIf
		
		\State $w1 \gets 3$
		\State $h1 \gets h-1$
		\State $x1 \gets x$
		\State $y1 \gets y+1$
		\If{$h1 > 0 \ \mathbf{and} \ y1<h$}
			\State \Call{Cover}{$B$,$w1$,$h1$,$x1$,$y1$,$r+1$}
		\EndIf
		
		\State $w2 \gets w-3$
		\State $h2 \gets h$
		\State $x2 \gets x+3$
		\State $y1 \gets y$
		
		\If{$w2 > 0 \ \mathbf{and} \ x2<w$}
		\State \Call{Cover}{$B$,$w2$,$h2$,$x2$,$y2$,$r+1$}
		\EndIf
		
		\EndProcedure
	\end{algorithmic}
\end{algorithm}

该算法的结果应当是$B$被填满,填上相同数字$i$的三连部分表示在递归树的第$i$层,这些位置被铺上了三连格。更新参数$r$的部分可以修改,使得填入代表不同含义的数字。

\clearpage

\section{XOJ 1004}

本题要求最大公约数和最小公倍数,由公式

$$
gcd(x,y)=gcd(b,a\mod b) \ \ \ \ a>b\mbox{且}a\mod b \ne 0
$$

和$gcd$与$lcm$的关系

$$
lcm(x,y)=x \cdot y \cdot gcd(x,y)
$$

可以简单地给出递归实现代码\ref{code:gcd},运行结果如图\ref{fig:xoj1004}

\begin{lstlisting}[language=Python,numbers=left,style=PythonStyle,caption=XOJ1004,label={code:gcd}]
def gcd(a, b):
	return b if a % b == 0 else gcd(b, a % b)


a, b = map(int, input().split())
g = gcd(a, b)

print(g, a * b // g, sep='\n')
\end{lstlisting}

\begin{figure}[htbp]
	\centering
	\begin{minipage}[t]{0.48\textwidth}
		\centering
		\includegraphics[width=6cm]{t1.png}
		\caption{XOJ1004}
		\label{fig:xoj1004}
	\end{minipage}
	\begin{minipage}[t]{0.48\textwidth}
		\centering
		\includegraphics[width=6cm]{t2.png}
		\caption{XOJ1057}
		\label{fig:xoj1057}
	\end{minipage}
\end{figure}


\section{XOJ1057}

按题意在区间$[1,n]$内计算有几个数字被$n$整除即可,运行结果如图\ref{fig:xoj1057}

\begin{lstlisting}[language=Python,numbers=left,style=PythonStyle,caption=XOJ1057,label={code:multipier}]
n = int(input())

print(sum(1 for i in filter(lambda x: n % x == 0, range(1, n + 1))))
\end{lstlisting}

\end{spacing}
\end{document}
\documentclass[a4paper,twoside]{article}
\usepackage{blindtext}  
\usepackage{geometry}

% Chinese support
\usepackage[UTF8, scheme = plain]{ctex}

% Page margin layout
\geometry{left=2.3cm,right=2cm,top=2.5cm,bottom=2.0cm}


\usepackage{listings}
\usepackage{xcolor}
\usepackage{geometry}
\usepackage{amsmath}
\usepackage{float}
\usepackage{hyperref}

\usepackage{graphics}
\usepackage{graphicx}
\usepackage{subfigure}
\usepackage{epsfig}
\usepackage{float}

\usepackage{algorithm}
\usepackage[noend]{algpseudocode}

\usepackage{booktabs}
\usepackage{threeparttable}
\usepackage{longtable}
\usepackage{listings}
\usepackage{tikz}
\usepackage{multicol}

% cite package, to clean up citations in the main text. Do not remove.
\usepackage{cite}

\usepackage{color,xcolor}

%% The amssymb package provides various useful mathematical symbols
\usepackage{amssymb}
%% The amsthm package provides extended theorem environments
\usepackage{amsthm}
\usepackage{amsfonts}
\usepackage{enumerate}
\usepackage{enumitem}
\usepackage{listings}

\usepackage{indentfirst}
\setlength{\parindent}{2em} % Make two letter space in the first paragraph
\usepackage{setspace}
\linespread{1.5} % Line spacing setting
\usepackage{siunitx}
\setlength{\parskip}{0.5em} % Paragraph spacing setting

% \usepackage[contents =22920202204622, scale = 10, color = black, angle = 50, opacity = .10]{background}

\renewcommand{\figurename}{图}
\renewcommand{\lstlistingname}{代码} 
\renewcommand{\tablename}{表格}
\renewcommand{\contentsname}{目录}
\floatname{algorithm}{算法}

\graphicspath{ {images/} }

%%%%%%%%%%%%%
\newcommand{\StudentNumber}{22920202204622}  % Fill your student number here
\newcommand{\StudentName}{熊恪峥}  % Replace your name here
\newcommand{\PaperTitle}{作业(八)}  % Change your paper title here
\newcommand{\PaperType}{算法设计与分析} % Replace the type of your report here
\newcommand{\Date}{2022年4月20日}
\newcommand{\College}{信息学院}
\newcommand{\CourseName}{算法设计与分析}
%%%%%%%%%%%%%

%% Page header and footer setting
\usepackage{fancyhdr}
\usepackage{lastpage}
\pagestyle{fancy}
\fancyhf{}
% This requires the document to be twoside
\fancyhead[LO]{\texttt{\StudentName }}
\fancyhead[LE]{\texttt{\StudentNumber}}
\fancyhead[C]{\texttt{\PaperTitle }}
\fancyhead[R]{\texttt{第{\thepage}页,共\pageref*{LastPage}页}}


\title{\PaperTitle}
\author{\StudentName}
\date{\Date}

\lstset{
	basicstyle          =   \sffamily,          % 基本代码风格
	keywordstyle        =   \bfseries,          % 关键字风格
	commentstyle        =   \rmfamily\itshape,  % 注释的风格,斜体
	stringstyle         =   \ttfamily,  % 字符串风格
	flexiblecolumns,                % 别问为什么,加上这个
	numbers             =   left,   % 行号的位置在左边
	showspaces          =   false,  % 是否显示空格,显示了有点乱,所以不现实了
	numberstyle         =   \zihao{-5}\ttfamily,    % 行号的样式,小五号,tt等宽字体
	showstringspaces    =   false,
	captionpos          =   t,      % 这段代码的名字所呈现的位置,t指的是top上面
	frame               =   lrtb,   % 显示边框
}

\lstdefinestyle{PythonStyle}{
	language        =   Python, % 语言选Python
	basicstyle      =   \zihao{-5}\ttfamily,
	numberstyle     =   \zihao{-5}\ttfamily,
	keywordstyle    =   \color{blue},
	keywordstyle    =   [2] \color{teal},
	stringstyle     =   \color{magenta},
	commentstyle    =   \color{red}\ttfamily,
	breaklines      =   true,   % 自动换行,建议不要写太长的行
	columns         =   fixed,  % 如果不加这一句,字间距就不固定,很丑,必须加
	basewidth       =   0.5em,
}

\lstdefinestyle{CppStyle}{
	language        =   c++,
	basicstyle      =   \zihao{-5}\ttfamily,
	numberstyle     =   \zihao{-5}\ttfamily,
	keywordstyle    =   \color{blue},
	keywordstyle    =   [2] \color{teal},
	stringstyle     =   \color{magenta},
	commentstyle    =   \color{red}\ttfamily,
	breaklines      =   true,   % 自动换行,建议不要写太长的行
	columns         =   fixed,  % 如果不加这一句,字间距就不固定,很丑,必须加
	basewidth       =   0.5em,
}

\algnewcommand\algorithmicinput{\textbf{Input:}}
\algnewcommand\algorithmicoutput{\textbf{Output:}}
\algnewcommand\Input{\item[\algorithmicinput]}%
\algnewcommand\Output{\item[\algorithmicoutput]}%

\usetikzlibrary{positioning, shapes.geometric}

\begin{document}
	
%%%%%%%%%%%%%%%%%%%%%%%%%%%%%%%%%%%%%%%%%%%%
\makeatletter % change default title style
\renewcommand*\maketitle{%
	\begin{center} 
		\bfseries  % title 
		{\LARGE \@title \par}  % LARGE typesetting
		\vskip 1em  %  margin 1em
		{\global\let\author\@empty}  % no author information
		{\global\let\date\@empty}  % no date
		\thispagestyle{empty}   %  empty page style
	\end{center}%
	\setcounter{footnote}{0}%
}
\makeatother
%%%%%%%%%%%%%%%%%%%%%%%%%%%%%%%%%%%%%%%%%%%%
	
	
\thispagestyle{empty}

\vspace*{1cm}

\begin{figure}[h]
	\centering
	\includegraphics[width=4.0cm]{logo.png}
\end{figure}

\vspace*{1cm}

\begin{center}
	\Huge{\textbf{\PaperType}}
	
	\Large{\PaperTitle}
\end{center}

\vspace*{1cm}

\begin{table}[h]
	\centering	
	\begin{Large}
		\renewcommand{\arraystretch}{1.5}
		\begin{tabular}{p{3cm} p{5cm}<{\centering}}
			姓\qquad 名 & \StudentName  \\
			\hline
			学\qquad号 & \StudentNumber \\
			\hline
			日\qquad期 & \Date  \\
			\hline
			学\qquad院 & \College  \\
			\hline
			课程名称 & \CourseName  \\
			\hline
		\end{tabular}
	\end{Large}
\end{table}

\newpage

\title{
	\Large{\textcolor{black}{\PaperTitle}}
}
	
	
\maketitle
	
\tableofcontents
 
\newpage
\setcounter{page}{1}

\begin{spacing}{1.2}

\section{题9.2}

\textbf{源点:}教授家 

\textbf{汇点:}学校 

\textbf{顶点:}街道拐角

\textbf{边:}若$u$和$v$之间有街道,连一权值为1的边

则若存在一条权值$\ge 2$的流,就可以上同一所学校。

\section{题9.5}

\begin{proof}[证明:$c_f(u,v)+c_f(v,u)=c(u,v)+c(v,u)$]

由对称性$f(u,v)=f(v,u)$

\begin{equation*}
	\begin{aligned}
		c_f(u,v)+c_f(v,u)&=c(u,v)-f(u,v)+c(v,u)-f(v,u)\\
		&=c(u,v)-f(u,v)+c(v,u)-f(u,v)\\
		&=c(u,v)+c(v,u)
	\end{aligned}
\end{equation*}

\end{proof}

\section{题9.7}

\begin{proof}
	原命题等价于\emph{若一个网络中所有的容量之都不同,则存在一个唯一的最大流}

	则最大流$|f|$有
	\begin{equation*}
		\begin{aligned}
			|f|&=\mathop{\max}\sum_{v\in V}f(s,v)\\
			&=f(s,v_1)+f(s,v_2)+\cdots+(s,v_k)+\cdots+f(s,v_n)\\
		\end{aligned}
	\end{equation*}
	存在$v_k\in V$使得$f(s,v_k)=c(s,v_k)$

	若有两个最大流,设$f(s,v_k)=c(s,v_k)<f^{'}(s,v_k^{'})=c^{'}(s,v_k^{'})$
	则在f中用$(s,v_k^{'})$替代$(s,v_k)$可以使得$|f|$更大,且$|f^{'}|=|f|$
	则存在一个唯一的最大流。由最大流最小割定理得有唯一的最小割。
\end{proof}

\section{题9.8}

在剩余网络中从源点进行广度优先搜索,并记录路径,若到达汇点则找到一条增广路径。

\section{题9.9}

为了找到最大瓶颈容量的增广路,可以从源点到汇点进行深度优先搜索,然后选择瓶颈容量
最大的一条。如算法~\ref{algo:aug}

\begin{algorithm}
	\caption{找到最大瓶颈容量的增广路}
	\label{algo:aug}
	\begin{algorithmic}[1]
		\Input{有向无环图$G$,源点$s$,汇点$t$}
		\Procedure{Find}{$G,s,t$}

		\State $path=\emptyset$
		\State $length=0$

		\For{$u \in G.vertics$}
			\For{$v \in G.vertics$}
				\State $p,len=$ \Call{DFS}{$G$,$u$,$v$,$u$,$\emptyset$}
				\If{$len>length$}
					\State $path=p$
				\EndIf
			\EndFor
		\EndFor

		\State \Return $path$
		
		\EndProcedure

		\Procedure{DFS}{$G,s,t,c,len,p=\emptyset$}
			\If{$c==t$}
				\State \Return $p$
			\EndIf

			\For{$v \in G.vertics$}
				\If{$G.edge($c,$v) \ and \ not \ v.visited$}
					\State $v.visited=True$
					\State \Return \Call{DFS}{$G$,$s$,$t$,$\max(len,G.edge($c,$v).c)$,$v$,$p+v$}
				\EndIf
			\EndFor
		\EndProcedure
	\end{algorithmic}
\end{algorithm}	

\section{题9.18}

将信$i$与除$j$之外剩余的$n-1$个信封连边构成二分图,求该二分图的最大匹配,
就能尽可能多地正确装入信封中

\section{实现匈牙利算法}

\begin{lstlisting}[language=c++,numbers=left,style=CppStyle,caption={匈牙利算法},label={code:hgr}]
constexpr int NMAX = 510;
bool g[NMAX][NMAX]{}, vis[NMAX];
int p[NMAX]{};

void reset()
{
	memset(g, 0, sizeof(g));
	memset(vis, 0, sizeof(vis));
	memset(p, 0, sizeof(p));
}

bool match(int i, const int N)
{
	for (int j = 1; j <= N; j++)
	{
		if (g[i][j] && !vis[j])
		{
			vis[j] = true;
			if (!p[j] || match(p[j], N))
			{
				p[j] = i;
				return true;
			}
		}
	}
	return false;
}

int hungarian(const int M, const int N)
{
	int ret = 0;
	for (int i = 1; i <= M; i++)
	{
		memset(vis, 0, sizeof(vis));
		if (match(i, N))
		{
			ret++;
		}
	}
	return ret;
}
\end{lstlisting}

\section{实现SPA}

\begin{lstlisting}[language=c++,numbers=left,style=CppStyle,caption={匈牙利算法},label={code:hgr}]

int n, m, s, t, last[MAXN], flow[MAXN];
int bfs()
{
    memset(last, -1, sizeof(last));
    queue<int> q;
    q.push(s);
    flow[s] = INF;
    while (!q.empty())
    {
        int p = q.front();
        q.pop();
        if (p == t)
            break;
        for (int eg = head[p]; eg; eg = edges[eg].next)
        {
            int to = edges[eg].to, vol = edges[eg].w;
            if (vol > 0 && last[to] == -1) 
            {
                last[to] = eg;
                flow[to] = min(flow[p], vol);
                q.push(to);
            }
        }
    }
    return last[t] != -1;
}

int EK()
{
    int maxflow = 0;
    while (bfs())
    {
        maxflow += flow[t];
        for (int i = t; i != s; i = edges[last[i] ^ 1].to) 
        {
            edges[last[i]].w -= flow[t];
            edges[last[i] ^ 1].w += flow[t];
        }
    }
    return maxflow;
}
\end{lstlisting}

\end{spacing}

\end{document}